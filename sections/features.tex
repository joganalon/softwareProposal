\documentclass{article}

\author{Jonel C. Ganalon}
\date{19 March 2024}

\begin{document}

\section{Features}
%for students
%%cashless transaction
%%students can make their diet plan/indicate their diet
%%students can put reviews and comments for each food stalls
%%they can put which table they are ordering from
%%students can check the menus of each vendors without going back and forth
%%students can easily know the price of each food
\subsection{News Feed}
This is where the vendors can post important announcements regarding their store. Posts of the store that the customer follows will be shown here. This will be important for the customer to know the events, promos, new meals, discounts, and the like that their store offers. To avoid cluttering, students will not be able to upload posts here but they can only give comments.

\subsection{Vendors Directory}
The website will have a list of the vendors in the canteen. They can see the profiles of each vendors. They can also see the avaible meals they offer and the number of customers following the seller. The students can also see a number of students buying their specific meal. 

\subsection{Search and Meal Filter}
The student can search a specific food they want and related available meals will show up. There are also categories that can be used as filters. For example, the meals from each seller will be grouped by meat, vegetable, meat and vegetable, pasta, and the like.

\subsection{Popular Meal Rankings}
Each meal will have a number that describes the number of clients that have bought that meal. This will make it possible to rank the meals within a store, or among all the stores. Students can know the best selling meals from each store and the best selling meals among all the stores.

\subsection{Customer Feedback and Rating}
Students can give comments to meals or store to rate them or to give suggestion or recommendation. Through this, the sellers will be able to know the thoughts of their customers and they can improve their store to cater best to the preference of their customers.

\subsection{Student Profiles}
In order to make transactions and data, the customers will have their own accounts in the system. In here, they will have thier username and a description of their diet or preferred foods. The information here will be used for the creation of statistical data that will be beneficial to the sellers and university as a whole.

\subsection{Purchase History and Tracking}
Under the student profile, they can see their purchase history. This will able them to monitor their food intake and their health as a whole.

\subsection{Real-Time Seating Availability and Order Tracking}
A layout of the table within the canteen will be avaible in the website. This layout shows which table is vacant or occupied. The system will have colour coding that describes the time after an order came from a specific table. With this, the students can estimate which table in the canteen is vacant or occupied. The color updates based from the order of the customers. When making an order, the customer, will need to specify which table they are in. This means that table in the canteen will have designated numbers. 

\subsection{Dynamic Inventory Management}


%%they can see the statistics regarding their menu
%%they can make special offers during school events
%%they can provide discounts or privelege to their loyal customer
%%they can know from which table the order is coming from

\end{document}
