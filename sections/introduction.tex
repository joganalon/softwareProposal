\documentclass{article}

\author{Jonel C. Ganalon}
\date{19 March 2024}

\begin{document}

\section{Introduction}
%describe canteen in universities
%everyone has smartphones and the increasing usage of cashless payment
the school canteen should reflect the educational goals of the school and support and complement student learning.

There are already several softwares that automates the transaction between the seller and the students through cashless payments. Furthermore, there are also softwares that able students to order food remotely. However, orders can still pile up once students slack in taking their order on time. The ACMS provides solution to this.

\subsection{Project Context}
%describe the status quo in Bonoan
The Bonoan building has its first floor and second floor as the canteen area of the ateneo students, not just for the college but also for the senior high school students.
During lunch time, students gather in the Bonoan building to eat their meals and also catch up with their friends. It is a common sight that this place is crowded especially at the traffic area front of the stores. There are different reasons why students crowd this area. First, students are looking at the menus deciding which meal to buy. Second, students are waiting for their meal. Third, students suddenly saw an acquaintant and started to catch up in the middle of the pathway. Fourth, students are paying or recieving their change. These become wors as the number of studens in the Ateneo grow. Take in mind that it is not just the college students that go here but also a number of senior high school students. A system to lessen the congestion in places like the canteen is by having staggered lunch times. However, this solution will not be effective as the number of students increase. If the canteen are congested, this can increase the time of students getting their food and decrease their time for enjoyable eating.

%we'll describe the layout of the canteen here or maybe in the feature sec instead, we'll see
%we also want to give sense of privacy when eating, it is uncomfortable to eat with crowd of people passing back and forth

%it is said that crowdedness attributes to positive traits in a store, we will mention that and contend that this can be achieve through the online review system, or a number of how many students are purchasing in this specific store HAHA

%%for students
%%%crowds are formed in front of food stalls because different students decide what food to buy, waiting for their food, paying for their food, chitchatting with their friends, students asking the price and ingredients

%%%students food intake are limited by the available food in canteen. students cannot suggest food

%%%miscommunication between student and vendors

%%for vendors
%%%food waste.
%%%knowing the students. they can increase thier sale by making sure that they re serving what the students wants or in their diet. catering to the student preferences and dietary needs



%%for university
%%%they are not aware of the transactions in the canteen
%%%they can fully contribute to the sustainability goals of UN by lesseing food waste
%%%the number of studetns are growing, transitioning to automated operations will be necessary

\subsection{Purpose and Description}
%we will divide it as well into three:students, vendors, and university.
This project aims to modernise the transactions inside the canteen at the Bonoan building of Ateneo de Naga University. This project aims to give benefits to the AdNU students, canteen vendors, and the university itself as a whole.
%stuents
%%ease the transaction of buying food
%%lessen the time ordering for food
%%provide them healthier food choices
%%by giving them power to decide what food they should eat

%vendors


\subsection{Objectives}
%what we want to achieve like a goal or tasks

\subsection{Scope and Limitations}
This project is to be used at the canteen of Ateneo de Naga University by its students, and teaching and non-teaching personnels. Specifically, this will be implemented at the Bonoan building. 
%only in the bonoan in AdNU


\end{document}
