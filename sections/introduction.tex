\documentclass{article}

\author{Jonel C. Ganalon}
\date{19 March 2024}

\begin{document}

\section{Introduction}
%describe canteen in universities



\subsection{Project Context}
%describe the status quo in Bonoan

%%for students
%%%crowds are formed in front of food stalls because different students decide what food to buy, waiting for their food, paying for their food, chitchatting with their friends, students asking the price and ingredients

%%%students food intake are limited by the available food in canteen. students cannot suggest food 

%%for vendors
%%%food waste.
%%%knowing the students. they can increase thier sale by making sure that they re serving what the students wants or in their diet. catering to the student preferences and dietary needs


%%for university
%%%they are not aware of the transactions in the canteen
%%%they can fully contribute to the sustainability goals of UN by lesseing food waste
%%%the number of studetns are growing, transitioning to automated operations will be necessary

\subsection{Purpose and Description}
%we will divide it as well into three:students, vendors, and university.
%stuents
%%ease the transaction of buying food
%%lessen the time ordering for food
%%provide them healthier food choices
%%by giving them power to decide what food they should eat

%vendors


\subsection{Objectives}

\subsection{Scope and Limitations}



\end{document}
