\section{Introduction}
The canteen facility within Ateneo de Naga University holds significant importance, serving not only as a place for students and employees to purchase meals but also as a social hub where they can gather, eat, and interact with peers. Given its central role, it's crucial that the canteen offers healthy meal options to support students' academic performance and overall well-being. Moreover, creating a conducive environment within the canteen is essential for fostering a positive learning atmosphere that aligns with the educational goals of the institution.

With the widespread use of smartphones among students, particularly accelerated by the shift to online learning during the Covid-19 pandemic, there's a unique opportunity to leverage technology to enhance the canteen experience. While existing software solutions facilitate cashless transactions and remote ordering, challenges such as order delays persist. The proposed ACMS (Automated Canteen Management System) aims to address these issues by streamlining operations and improving efficiency.

\subsection{Project Context}
The Bonoan building serves as the primary canteen area for Ateneo de Naga University students, accommodating both college and senior high school students during lunchtime. However, the space often becomes overcrowded, leading to congestion and inefficiencies in the food purchasing process. Factors contributing to this congestion include students browsing menus, waiting for their orders, engaging in social interactions, and completing transactions. The growing student population exacerbates these challenges, highlighting the need for a systematic solution. Despite being a public space, the canteen should provide students with a sense of privacy and comfort while dining, which can be compromised during peak hours. Additionally, limited menu visibility and potential miscommunications between vendors and customers further impede the overall dining experience. These issues not only affect students but also impact vendors and the university administration in terms of resource management and sustainability goals.

\subsection{Purpose and Description}
The primary goal of the ACMS project is to modernize and optimize the canteen operations within the Bonoan building of Ateneo de Naga University. This initiative aims to benefit the customers, vendors, and the university as a whole by improving transaction efficiency, reducing wait times, enhancing food choices, and promoting healthier eating habits.

\subsubsection{For Customers}
"Customers" refer to individuals affiliated with Ateneo de Naga University, including students and staff members, who utilize the canteen services for purchasing meals and other related transactions.  The ACMS project impacts the customers in the following ways:
\begin{itemize}
\item Streamline food purchasing transactions
\item Reduce wait times for ordering and receiving meals
\item Expand access to healthier food options
\item Empower students to make informed dietary choices
\end{itemize}

\subsubsection{For Vendors}
"Vendors" pertain to the various food establishments operating within the canteen premises of the Bonoan building at Ateneo de Naga University. These vendors offer a range of food options to customers and engage in transactions within the canteen environment. The ACMS project impacts the vendors in the following ways:
\begin{itemize}
\item Provide an efficient platform for managing transactions
\item Minimize food wastage through data-driven insights
\item Cater to student preferences and dietary needs to optimize sales
\end{itemize}

\subsubsection{For University}
"University" refers to Ateneo de Naga University as an institution responsible for overseeing the canteen operations and supporting initiatives to enhance the student experience. The ACMS project impacts the university in the following ways:
\begin{itemize}
\item Enhance awareness and oversight of canteen transactions
\item Contribute to sustainability goals by reducing food waste
\item Adapt to the growing student population through automated operations
\end{itemize}

\subsection{Objectives}
The objectives of the ACMS project include:
\begin{itemize}
\item Implementing a user-friendly digital platform for canteen transactions
\item  Enhancing menu visibility and customization options for students
\item  Improving communication and coordination between vendors and customers
\item  Providing data analytics capabilities to optimize resource allocation and reduce waste
\end{itemize}

\subsection{Scope and Limitations}
The ACMS project will be specifically implemented within the canteen of Ateneo de Naga University's Bonoan building, targeting students, faculty, and staff. While the system aims to address various inefficiencies within the canteen operations, it may have limitations in terms of scalability and compatibility with existing infrastructure. Additionally, the scope of the project may not encompass broader issues related to food quality, pricing, or external factors affecting the dining experience.
